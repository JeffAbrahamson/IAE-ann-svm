\input ../notes-header.tex
\usepackage{epsfig}
\usepackage{bbm}

\begin{document}




\notetitle{Support Vector Machines}



%%%%%%%%%%%%%%%%%%%%%%%%%%%%%%%%%%%%%%%%%%%%%%%%%%
\notetitle{Lagrange Multipliers}

We've talked about optimisation, but life often supplies constraints.

Example: Maximise production but there's a budget

Example: Maximise fun but there's a budget

Example: Live as long as you can on Mars but you have minimum daily
calorie requirements to survive

\bigskip

Suppose we have a function $f : \mathbbm{R}^n \rightarrow
\mathbbm{R}$ that we want to maximise.

Then look at $\{x\mid \nabla f(x) = 0 \}$.

\medskip

Suppose moreover that we have another function $g : \mathbbm{R}^n
\rightarrow \mathbbm{R}$ and that we have some constraint like
$g(x)=0$.

So
\begin{displaymath}
  \mbox{Maximize } f(x,y) \mbox{ subject to } g(x,y) = 0
\end{displaymath}

\textbf{Example:}

\begin{equation}
\begin{cases}
  f(x,y) & = xy \\
  g(x,y) & = x^2 + y^2 = 1
\end{cases}
\end{equation}

How do we approach this?

Consider isoclines $f(x,y) = c$.

And consider isoclines $g(x,y) = c'$ (not necessarily the same
constant).

We want tangent points.

Recall that gradient is $\bot$ to isoclines.\\
So we want the points where the $\nabla$ are $\parallel$.

So we want set of
\begin{displaymath}
  \{ x \mid \nabla f(x) = \lambda \nabla g(x) \}
\end{displaymath}

\textbf{Example:}

\begin{equation}
  \begin{cases}
    \nabla f & = 
    \begin{pmatrix}
      y \\ x
    \end{pmatrix}
    \\[8mm]
    \nabla g & =
    \begin{pmatrix}
      2x \\ 2y
    \end{pmatrix}
  \end{cases}
\end{equation}

So
\begin{equation}
  \begin{cases}
    y & = \lambda 2x
    \\
    x & = \lambda 2y
    \\
    1 & = x^2 + y^2
  \end{cases}
\end{equation}

From the first of these we get
\begin{displaymath}
  \lambda = \frac{y}{2x} \quad \mbox{ if } x \neq 0
\end{displaymath}

Substituting into the second,
\begin{displaymath}
  x = \frac{y}{2x} \cdot 2y = \frac{y^2}{x} \Rightarrow x^2 = y^2
  \Rightarrow x = \pm y
\end{displaymath}

And then the third provides
\begin{displaymath}
  2x^2 = 1 \Rightarrow x^2 = \frac 12 \Rightarrow \boxed{x =
  \frac{\pm\sqrt{2}}{2} \wedge y = -x}
\end{displaymath}

\exercise{Consider $f(x,y) = x y^2$ subject to $2x^2 + 5y^2 = 2$.}

\exercise{Consider $f(x,y,z) = x^2 y^3 z$ subject to $2x^2 + 5y^2 = 2$.}


%%%%%%%%%%%%%%%%%%%%%%%%%%%%%%%%%%%%%%%%%%%%%%%%%%
\notetitle{Lagrangian}

We can write
\begin{displaymath}
  L = f - \lambda g
\end{displaymath}

Then $\nabla L = 0$ is the set of equations we started with, above.

This is generally a nice form for computers.

It also transforms the problem from a constrained optimisation
problem into an unconstrained optimisation problem.


%%%%%%%%%%%%%%%%%%%%%%%%%%%%%%%%%%%%%%%%%%%%%%%%%%
\notetitle{Decision boundaries}

Consider these points.  What would be the decision boundaries for
\begin{itemize}
\item logistic regression
\item CART
\item ANN
\end{itemize}

\centerline{\input{decision.pdf_t}}

\bigskip

Consider a sort of ``ideal'' decision boundary.  Stay linear, but
which line?

Concept of widest street (or: bicycle lane).

How do we make a decision boundary that will use that line?

\bigskip
\centerline{\input{decision-svm1.pdf_t}}
\bigskip

Consider a vector $w$ that is $\bot$ to the separator that we want.
(We don't know its length.)

Let $u$ be a vector pointing to some unknown point that we want to
classify.

Is $u$ on the $+$ or the $-$ side of the decision boundary (zone, margin)?

Let's project $u$ onto $w$.  Then we want
\begin{displaymath}
  w\cdot u \ge c
\end{displaymath}

which is the same as saying this, which we'll call our decision rule:
\begin{equation}
  \label{eq:decision-rule}
  \boxed{w\cdot u + b \ge 0 \Rightarrow +}
\end{equation}

But we don't know $b$ or $w$.

We want to add contraints so that we can calculate them.

Pick positive sample point $x_+$, negative sample point $x_-$.

Want
\begin{align*}
  w\cdot x_+ + b & \ge 1
  \\
  w\cdot x_- + b & \le -1
\end{align*}
Note that the $\pm 1$ indicates the margin.  Zero is the decision
boundary (the middle line).

For convenience, introduce
\begin{equation*}
  y_i =
  \begin{cases}
    1 & \mbox{positive samples} \\
    -1 & \mbox{negative samples}
  \end{cases}
\end{equation*}

Multiplying by $y_i$, we get
\begin{align*}
  y_i(w\cdot x_+ + b) & \ge 1
  \\
  y_i(w\cdot x_- + b) & \ge 1
\end{align*}

These are the same!
\begin{displaymath}
  y_i(w\cdot x_i + b) - 1 \ge 0
\end{displaymath}

So for $x_i$ on the margin, we have
\begin{equation}
  \label{eq:on-margin}
  \boxed{y_i(w\cdot x_i + b) - 1 = 0}
\end{equation}

We want to maximise the margin.  But what is its width?

If we have a unit vector, we could set the width.  But we can subtract $x_+-x_-$:
\begin{equation}
  \label{eq:width-raw}
  \mbox{width} = (x_+ - x_-) \cdot \frac{w}{\parallel w\parallel}
\end{equation}

But we have
\begin{displaymath}
  1(w\cdot x_+ + b) - 1 = 0 \iff w\cdot x_+ = 1-b
\end{displaymath}
and
\begin{displaymath}
  -1(w\cdot x_- + b) - 1 = 0 \iff -w\cdot x_- = 1+b
\end{displaymath}

So combining with eq. (\ref{eq:width-raw}), we get
\begin{equation}
  \label{eq:width}
  \mbox{width} = \frac{2}{\parallel w\parallel}
\end{equation}

We want to maximise (\ref{eq:width}).

So we can maximise $\frac{1}{\parallel w\parallel}$ $\Rightarrow$
minimise $\parallel w\parallel$ $\Rightarrow$
minimize $\;\frac 12 \parallel w\parallel^2$.

\bigskip

Let's use the Lagrangian!

\begin{equation}
  \label{eq:lagrangian}
  L = \frac 12 \parallel w\parallel^2 - \sum \lambda_i \left[ y_i(w\cdot
    x_i + b) - 1 \right]
\end{equation}

(It will turn out that $\lambda$ is non-zero only on the margin.)

\begin{displaymath}
  \frac{\partial L}{\partial w} = w - \sum\lambda y_i x_i = 0
\end{displaymath}
and so
\begin{equation}
  \label{eq:w}
  w = \sum\lambda_i y_i x_i
\end{equation}

So $w$ is a linear combination of the samples!

In addition,
\begin{displaymath}
  \frac{\partial L}{\partial b} = -\sum \lambda_i y_i = 0
\end{displaymath}
and so
\begin{equation}
  \label{eq:b}
  \sum \lambda_i y_i = 0
\end{equation}

So what happens to $L$ (eq. \ref{eq:lagrangian})?

\bigskip

\begin{equation}
  \label{eq:learning-rule}
  L = \frac 12 \left( \sum \lambda_i y_i x_i\right)
  \left(\sum \lambda_j y_j x_j\right)
  -
  \sum \lambda_i y_i x_i
  \left(\sum \lambda_j y_j x_j\right)
  - \boxed{\sum \lambda_i y_i} b + \sum \lambda_i
\end{equation}
The boxed portion is zero, so this leaves us with
\begin{equation}
  \label{eq:recognition-rule}
  \sum\lambda_i - \frac 12 \sum_i \sum_j \lambda_i \lambda_j y_i y_j
  x_i\cdot x_j
\end{equation}

This is the optimisation: the part where we learn.

So the optimisation depends on on the dot products of pairs of
samples!

\bigskip

Let's go back to our decision rule, eq. (\ref{eq:decision-rule}).

\begin{displaymath}
  \sum \lambda_i y_i x_i\cdot u + b \ge 0 \Rightarrow +
\end{displaymath}
Thi is the the decision boundary: it only depends on the sample vector
and unknown!

It can be shown that this is a convex space, so we can't get stuck at
local maxima.

We've seen that we can learn and decide using only dot products among
sample data points and between sample data points and unknown (to be
classified) points.  This is the property that lets us use kernels.


%%%%%%%%%%%%%%%%%%%%%%%%%%%%%%%%%%%%%%%%%%%%%%%%%%
\notetitle{Kernels}

But what if our data is not linearly separable?  (Example: XOR.)

The optimisation only depends on dot products, so what if we had a
function $\phi$ that mapped our problem into a higher dimensional
space where it is linearly separable?

The learning rule (\ref{eq:learning-rule}) only depends on dot
products.  So we can map using $\phi$ and optimise in the new space.

In addition, the recognition rule (\ref{eq:recognition-rule}) also
only depends on dot product.  So we can recognise in the new space,
too.

In addition, we just need to define a function
\begin{displaymath}
  k(x_i, x_j) = \phi(x_i)\cdot\phi(y_i)
\end{displaymath}
and we don't actually need to know $\phi$.



%%%%%%%%%%%%%%%%%%%%%%%%%%%%%%%%%%%%%%%%%%%%%%%%%%
\notetitle{Popular Kernels}

Linear kernel:
$(u\cdot v + 1)$

Polynomial kernel:
$(u\cdot v + 1)^n$

Radial basis function (RBK, RBF):
\begin{displaymath}
  e^{(-\parallel x_i - x_j\parallel) / \sigma}
\end{displaymath}
(Note: If $\sigma$ is too small, we overfit.)


%%%%%%%%%%%%%%%%%%%%%%%%%%%%%%%%%%%%%%%%%%%%%%%%%%
\notetitle{History}

Vapnick did his PhD on this at Moscow University in the early 1960's.
He worked at an oncology institute.  He didn't have access to
computers, so this was mostly of theoretical interest.  No one in the
West knew about his work.

Someone at Bell Labs discovered him, invited him to visit.  He decided
to emigrate to the U.S. in 1991.

In 1992 he submitted three papers to NIPS.  All are rejected.

In 1993--1994, Bell Labs becomes interested in OCR and ANNs.  Vapnick
thinks ANNs suck, bets a colleague a good dinner that SVM will do a
better job than an ANN.

Colleague tries linear kernel with n=2, and it works.  This was the
first use of kernels.  (It was in Vapnick's PhD thesis, but he didn't
think it was very important.)

So it took 30 years from his PhD (understanding the problem) to
appreciating the importance of the technique.


%%%%%%%%%%%%%%%%%%%%%%%%%%%%%%%%%%%%%%%%%%%%%%%%%%
\notetitle{Soft Margin}


\end{document}
