\input ../notes-header.tex
\usepackage{epsfig}
\usepackage{bbm}

\begin{document}

\notetitle{Supplement: Optimising under Constraint}



%%%%%%%%%%%%%%%%%%%%%%%%%%%%%%%%%%%%%%%%%%%%%%%%%%
\notetitle{Lagrange Multipliers}

We've talked about optimisation, but life often supplies constraints.

Example: Maximise production but there's a budget

Example: Maximise fun but there's a budget

Example: Live as long as you can on Mars but you have minimum daily
calorie requirements to survive

\bigskip

Suppose we have a function $f : \mathbbm{R}^n \rightarrow
\mathbbm{R}$ that we want to maximise.

Then look at $\{x\mid \nabla f(x) = 0 \}$.

\medskip

Suppose moreover that we have another function $g : \mathbbm{R}^n
\rightarrow \mathbbm{R}$ and that we have some constraint like
$g(x)=0$.

So
\begin{displaymath}
  \mbox{Maximize } f(x,y) \mbox{ subject to } g(x,y) = 0
\end{displaymath}

\textbf{Example:}

\begin{equation}
\begin{cases}
  f(x,y) & = xy \\
  g(x,y) & = x^2 + y^2 = 1
\end{cases}
\end{equation}

How do we approach this?

Consider isoclines $f(x,y) = c$.

And consider isoclines $g(x,y) = c'$ (not necessarily the same
constant).

We want tangent points.

Recall that gradient is $\bot$ to isoclines.\\
So we want the points where the $\nabla$ are $\parallel$.

So we want set of
\begin{displaymath}
  \{ x \mid \nabla f(x) = \lambda \nabla g(x) \}
\end{displaymath}

\textbf{Example:}

\begin{equation}
  \begin{cases}
    \nabla f & = 
    \begin{pmatrix}
      y \\ x
    \end{pmatrix}
    \\[8mm]
    \nabla g & =
    \begin{pmatrix}
      2x \\ 2y
    \end{pmatrix}
  \end{cases}
\end{equation}

So
\begin{equation}
  \begin{cases}
    y & = \lambda 2x
    \\
    x & = \lambda 2y
    \\
    1 & = x^2 + y^2
  \end{cases}
\end{equation}

From the first of these we get
\begin{displaymath}
  \lambda = \frac{y}{2x} \quad \mbox{ if } x \neq 0
\end{displaymath}

Substituting into the second,
\begin{displaymath}
  x = \frac{y}{2x} \cdot 2y = \frac{y^2}{x} \Rightarrow x^2 = y^2
  \Rightarrow x = \pm y
\end{displaymath}

And then the third provides
\begin{displaymath}
  2x^2 = 1 \Rightarrow x^2 = \frac 12 \Rightarrow \boxed{x =
  \frac{\pm\sqrt{2}}{2} \wedge y = -x}
\end{displaymath}

\exercise{Consider $f(x,y) = x y^2$ subject to $2x^2 + 5y^2 = 2$.}

\exercise{Consider $f(x,y,z) = x^2 y^3 z$ subject to $2x^2 + 5y^2 = 2$.}


%%%%%%%%%%%%%%%%%%%%%%%%%%%%%%%%%%%%%%%%%%%%%%%%%%
\notetitle{Lagrangian}

We can write
\begin{displaymath}
  L = f - \lambda g
\end{displaymath}

Then $\nabla L = 0$ is the set of equations we started with, above.

This is generally a nice form for computers.

It also transforms the problem from a constrained optimisation
problem into an unconstrained optimisation problem.


\end{document}
