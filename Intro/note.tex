\input ../notes-header.tex

\begin{document}

\notetitle{Introduction}

My name.  My email.

\begin{itemize}
\item Machine learning
\item Data science
\item Artificial intelligence
\item Big data
\end{itemize}

A better title would be ``Fun with Separators''.\\
``Avoir du plaisir avec des séparateurs''.

\section*{What we'll learn}

\begin{itemize}
\item SVM
\item ANN
\end{itemize}

Note that ANN is not ``more advanced'' just because it's second.  It's
just harder.

Things you know and relationships to them:
\begin{itemize}
\item Logistic regression
\item CARTs
\item Random forests
\item Also, these techniques are usable for regression
\end{itemize}


\section*{Format}

There will be weekly assignments.  We will discuss them in class but I
won't correct them unless you ask.

There will be a very short quiz at the beginning of each session.  It
will start and end on time.  It will be easy if you did the
assignment.  If you didn't, it will probably be impossible.
(Sometimes it will be writing code.)

(Olivier Darné has asked for the results.)

You will have a final project, which you may do together or
individually.  There will be a 15 minute oral examination, possibly by
video, in which you will be asked some nitty questions to determine
that you understand what you did and why.

You should probably plan to spend 2--3 hours outside of class for each
hour in class.  There is, historically, a very strong correlation
between your preparation time and your success.


\section*{github}

There's a github repository.

Recommend use your own laptop if possible.  Python and other software
is free and easy to install.  There's a help page in the git for
installing python.

You should create a github repository for the course.  \textit{Talk
  about how to ask questions.}  Code questions must be via github
issues.  (Tag me.)  Non-code questions may be by email or github as
seems most appropriate, but github issues are pretty convenient.

\end{document}
