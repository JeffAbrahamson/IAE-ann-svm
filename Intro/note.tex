\input ../notes-header.tex

\begin{document}

\notetitle{Introduction}

My name.  My email.

Learning to separate.  So learning why and when.

\section*{What we'll learn}

Things you know and relationships to them:
\begin{itemize}
\item Regression (minimise $\parallel Xb^T-y\parallel$)
\item Logistic regression --- think about naive bayesian classifiers
  --- $\pr(x\mid y)$ --- and then use the (logistic) classifier and
  learn it directly from the data --- $\pr(y\mid x)$
\item CARTs
\item Random forests
\item (CARTs and RF are also usable for regression.)
\end{itemize}

I explain what these things are in high-level terms:

\begin{itemize}
\item Logistic Regression
\item SVM
\item ANN
\end{itemize}

Note that ANN is not ``more advanced'' just because it's second.  It's
just harder.  And, often, a more efficient way to make time pass
without better results.


\section*{Format}

There will be weekly assignments.  We will discuss them in class but I
won't correct them unless you ask.

You will have a final project, which you must do in groups of two (or
three in at most one case).  The final exam will ask you some
questions about your project (individually).

You should probably plan to spend 2--3 hours outside of class for each
hour in class.  There is, historically, a very strong correlation
between your preparation time and your success.

(Discussion based on README here.)

\section*{github}

There's a github repository.

Recommend use your own laptop if possible.  Python and other software
is free and easy to install.  There's a help page in the git for
installing python.

You should create a github repository for the course.  \textit{Talk
  about how to ask questions.}  Code questions must be via github
issues.  (Tag me.)  Non-code questions may be by email or github as
seems most appropriate, but github issues are pretty convenient.

\end{document}
