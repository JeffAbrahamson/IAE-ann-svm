\input ../notes-header.tex

\begin{document}

\notetitle{Logistic Regression}

Remember conditional probability?

\begin{defn}[Kolmogorov]
  \begin{displaymath}
    \pr(A\mid B) = \frac{\pr(A\cap B)}{\pr(B)}
  \end{displaymath}
\end{defn}

\begin{defn}[axiom]
  \begin{displaymath}
    \pr(A\cap B) = \pr(A\mid B) \, \pr(B)
  \end{displaymath}
\end{defn}

\begin{example}[mostly reliable]
  Suppose a specific test protocol has a 20\% false negative rate and
  a 1\% false positive rate.  If you test positive, what is the
  probability you are positive?

  Suppose first that the incident rate is 50\%.  That it is 1\%.
\end{example}

\begin{soln}
    Show with measuring squares, then with (Kolmogorov) definition.
\end{soln}

\begin{example}[less reliable]
  Suppose a specific test protocol has a 10\% false negative rate and
  a 50\% false positive rate.  If you test positive, what is the
  probability you are positive?

  Suppose first that the incident rate is 50\%.  That it is 1\%.
\end{example}

\begin{soln}
  Show with measuring squares, then with (Kolmogorov) definition.
\end{soln}

\begin{example}[generating points]
  Given a point and a gaussian distribution, what is the probability
  that a point is produced at a given location?  Within a given interal?
\end{example}

\begin{soln}
  Show with picture.
\end{soln}

\begin{example}[validating points]
  Given a set of points and a new point, what is the probability that
  the new point is part of that distribution?
\end{example}

\begin{soln}
  This is actually logistic regression.

  \begin{itemize}
  \item Is this obvious?
  \item Is logistic regression a linear model?  Why or why not?
  \end{itemize}

  \begin{theorem}{Bayes}
    \begin{displaymath}
      \pr(H\mid D) = \frac{\pr(D\mid H)\, \pr(H)}{\pr(D)}
    \end{displaymath}
  \end{theorem}
  \begin{proof}
    From axiom definition and symmetry.
  \end{proof}

  Linear model: $y = \beta x + \beta_0$.

  Logistic regression = learn $\pr(H\mid D)$ or at least $\pr(H\mid
  x\in D)$.

  Usually one now simply shows
  \begin{displaymath}
    \pr(y\mid x) = \frac{1}{1 + e^{-(\beta x + \beta_0)}}
  \end{displaymath}
  which is the inverse logit function, $\beta, \beta_0$ are to be
  learned.

  Recall: logit is $\sigma(x) = 1/(1+e^{-x})$ and is
  defined by
  \begin{displaymath}
    logit(p) = \sigma^{-1}(p) = \ln \left( \frac{p}{1-p} \right)
  \end{displaymath}

  \begin{example}
    Making bread during lockdown.\\
    $H$ is ``made a good loaf of bread''.\\
    $D$ is vector of technique details (what I did).\\
    Want $\pr(H\mid D)$, the posterior probability (\textit{probabilité à posteriori})

    This is our signal to look at Bayes theorem.
    \begin{displaymath}
      \pr(H\mid D) = \frac{\pr(D\mid H)\, \pr(H)}{\pr(D)}
    \end{displaymath}
    Recall that $\pr(D\mid H)$ is the likelihood (vrasemblance),
    $\pr(H)$ and $\pr(D)$ are the prior (or marginal) probabilities
    (\textit{probabilités à priori}).

    We also call $\pr(D)$ the evidence.

    $\pr(H)$ we observe.  But the others?  Especially $\pr(D)$ we have
    no clue.

    So we introduce $\pr(\overline{H})$.
    \begin{displaymath}
      \pr(\overline{H}\mid D) = \frac{\pr(D\mid \overline{H})\, \pr(\overline{H})}{\pr(D)}
    \end{displaymath}

    Divide, the $\pr(D)$'s cancel.  We'll call these odds
    (\textit{cote} in French).

    \begin{displaymath}
      C(H\mid D) = \frac{\pr(H\mid D)}{\pr(\overline{H}\mid D)}
      = \frac{\pr(D\mid H)\, \pr(H)}{\pr(D\mid \overline{H})\,
        \pr(\overline{H})}
      = \frac{\pr(D\mid H)}{\pr(D\mid \overline{H})} \, C(H)
    \end{displaymath}

    A linear model would look like $y = \beta x + \beta_0$.  We don't
    have that yet.  But we could take logs.

    \begin{displaymath}
      \ln(C(H\mid D)) =
      \ln\left( \frac{\pr(D\mid H)}{\pr(D\mid \overline{H})} \right)
      + \ln(C(H))
    \end{displaymath}
  \end{example}

  The first term we call the log likelihood.\\
  The second term is basically constant, the log prior.

  Assume the log likelihood is basically a linear function of $D$.
  Then we get
  \begin{displaymath}
    \ln(C(H\mid D)) = \beta D + \beta_0
  \end{displaymath}

  And we can learn $\beta, \beta_0$, then take exponents to get to the
  quantity $\pr(H\mid D)$ that we wanted at the start.
\end{soln}

\bigskip

There's more work to do.  We need to do some work to understand in
detail how to pass this to an optimiser, what the right loss function
is for gradient descent.  We're not going to do that here.

\begin{lemma}[Aside, only discuss in class if questioned.]
  We can recover $\pr(H)$ from $C(H)$.
\end{lemma}

\begin{proof}
  \begin{align*}
    C(H) & = \frac{\pr(H)}{\pr(\overline{H})} =
           \frac{\pr(H)}{1-\pr(H)} \\[2mm]
    \pr(H) & = (1-\pr(H)) C(H)  \\[2mm]
         & = \frac{1+C(H)}{1+C(H)}(1-\pr(H)) C(H) \\[2mm]
         & = \frac{\red{(1+C(H))(1-\pr(H))} C(H)}{1+C(H)} \\[2mm]
  \end{align*}

  \begin{align*}
    \red{(1+C(H))(1-\pr(H))} & = \left(1+\frac{\pr(H)}{\pr(\overline
                               H)}\right) \left(1-\pr(H)\right) \\[2mm]
                             &= \frac{
                               \pr(\overline H)
                               + \pr(H)
                               - \pr(H)
                               - \pr(H)^2}
                               {\pr(\overline H)} \\[2mm]
                             &= \frac{\pr(\overline H) + \pr(H) -
                               \pr(H)(\pr(\overline H) + \pr(H))}
                               {\pr(\overline H)} \\[2mm]
                             &= \frac{1-\pr(H)}{\pr(\overline H)} \\[2mm]
                             &= \frac{1-\pr(H)}{1-\pr(H)} \\[2mm]
                             &= 1
  \end{align*}
\end{proof}

\bigskip

Review of this review:
\begin{itemize}
\item Logistic regression is a linear model.
\item It's based on Bayes Theorem.  It gives us probabilities based on
  observed data.
\item It's simple to interpret.
\end{itemize}

The point of this course is going to be to understand two other
separation algorithms: SVM and ANN.

Both have the property that they can be linear or transform their
problem to find non-linear separators.

One of the goals of this course is that you \textbf{don't} think of your job as
trying things until one works.

\textbf{SVM}

Quick introduction to the motivation for SVM: maximum margins.

Reminder for next time: projects, git repos.  Test your python
installations (or the ones the school provides).

\end{document}
