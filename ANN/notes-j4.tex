\input ../notes-header.tex
\usepackage{epsfig}

\begin{document}


%%%%%%%%%%%%%%%%%%%%%%%%%%%%%%%%%%%%%%%%%%%%%%%%%%%%%%%%%%%%
\notetitle{Rules of Machine Learning}

(source: Martin Zinkevich, Google)

\notesection{Some terms}

\begin{itemize}
\item \textbf{Instance}: The thing about which you want to make a
  prediction. For example, the instance might be a web page that you
  want to classify as either "about cats" or "not about cats".
\item \textbf{Label}: An answer for a prediction task ­­ either the
  answer produced by a machine learning system, or the right answer
  supplied in training data. For example, the label for a web page
  might be "about cats".
\item \textbf{Feature}: A property of an instance used in a prediction
  task. For example, a web page might have a feature "contains the
  word 'cat'".
\item \textbf{Feature Column:} A set of related features, such as the
  set of all possible countries in which users might live. An example
  may have one or more features present in a feature column. "Feature
  column" is Google-specific terminology. A feature column is referred
  to as a "namespace" in the VW system (at Yahoo/Microsoft), or a
  field.
\item \textbf{Example:} An instance (with its features) and a label.
\item \textbf{Model:} A statistical representation of a prediction
  task. You train a model on examples then use the model to make
  predictions.
\item \textbf{Metric:} A number that you care about. May or may not be
  directly optimized.
\item \textbf{Objective:} A metric that your algorithm is trying to optimize.
\item \textbf{Pipeline:} The infrastructure surrounding a machine
  learning algorithm. Includes gathering the data from the front end,
  putting it into training data files, training one or more models,
  and exporting the models to production.
\item \textbf{Click-through Rate:} The percentage of visitors to a web
    page who click a link in an ad.
\end{itemize}


\notesection{Overview}

\begin{itemize}
\item Engineering is more important than ML.  If it's not reliable,
  solid, and reproducible, the rest doesn't matter.
\item Have reasonable objects
\item Be as simple as possible
\end{itemize}


\notesection{Start without ML}

\begin{itemize}
\item ML needs data, you rarely start with lots of data
\item Heuristics will often get you half way there
\item Your first goal is just to be better than random.  So identify
  what random looks like.
\end{itemize}

\notesection{Design and Implement Metrics}

\begin{itemize}
\item Start by measuring, otherwise you can't know how you're doing
\item Measuring the first thing is the hardest
\item People care less early, so less resistance
\item Get historical data now.  When you start to care, you'll have a baseline.
\end{itemize}

\notesection{Prefer ML to complex heuristics}

\begin{itemize}
\item It's more maintainable
\item But have you tried simple heuristics?
\end{itemize}

\notesection{Start with simple models and get infrastructure right}

\begin{itemize}
\item If your pipeline is shoddy, it will be hard to do anything
  anyway
\item The first model provides the biggest delta
\item This is ``hello world'' territory, focus on the basics
  \begin{itemize}
  \item getting data
  \item representing data
  \item identifying good vs bad
  \item how to integrate model into application
  \end{itemize}
\item Simple features are easier to understand, debug
\item Make sure you understand your data
\end{itemize}

\notesection{Test infrastructure separately from ML}

\begin{itemize}
\item Make sure the infra is testable
\item Make sure the ML is encapsulated
\item Test getting data into the system
\item Test that features are populated correctly
\item Inspect the data (if allowed)
\item Compare statistics from your pipeline with other sources (if
  exist)
\item Test moving models from training to production
\item Make sure you understand your data
\end{itemize}

\notesection{Heuristics become features}

\begin{itemize}
\item Often some system already exists.  It uses heuristics, produces
  features.  Take advantage of that.
\item Consider using the existing system as a sort of pre-processor,
  generating synthetic features.
\end{itemize}

\notesection{Monitoring and alerting are important}

\begin{itemize}
\item Understand your freshness requirements
\item Do sanity checks at model export time, at deploy time
\item Understand what requires an email, what requires a page, what
  just has to be available for inspection
\item Be aware of silent failures (e.g., data source decay)
\item Make sure features have owners and that it's documented who they
  are (and that the features are documented).  Same for algorithms.
\end{itemize}

\end{document}
